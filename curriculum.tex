%%%%%%%%%%%%%%%%%%%%%%%%%%%%%%%%%%%%%%%%%%%%%%%%%%%%%%%%%%%%%%%%%%%%%%
% LaTeX Template: Curriculum Vitae
%
% Source: http://www.howtotex.com/
% Feel free to distribute this template, but please keep the
% referal to HowToTeX.com.
% Date: July 2011
%Version for spanish users, by dgarhdez
% 
%%%%%%%%%%%%%%%%%%%%%%%%%%%%%%%%%%%%%%%%%%%%%%%%%%%%%%%%%%%%%%%%%%%%%%
% How to use writeLaTeX: 
%
% You edit the source code here on the left, and the preview on the
% right shows you the result within a few seconds.
%
% Bookmark this page and share the URL with your co-authors. They can
% edit at the same time!
%
% You can upload figures, bibliographies, custom classes and
% styles using the files menu.
%
% If you're new to LaTeX, the wikibook is a great place to start:
% http://en.wikibooks.org/wiki/LaTeX
%
%%%%%%%%%%%%%%%%%%%%%%%%%%%%%%%%%%%%%%%%%%%%%%%%%%%%%%%%%%%%%%%%%%%%%%
\documentclass[paper=a4,fontsize=11pt]{scrartcl} % KOMA-article class
							
\usepackage[english]{babel}
\usepackage[utf8x]{inputenc}
\usepackage[protrusion=true,expansion=true]{microtype}
\usepackage{amsmath,amsfonts,amsthm}     % Math packages
\usepackage{graphicx}                    % Enable pdflatex
\usepackage[svgnames]{xcolor}            % Colors by their 'svgnames'
\usepackage{geometry}
	\textheight=700px                    % Saving trees ;-)
\usepackage{url}
\usepackage{float}
\usepackage{wrapfig}
\restylefloat{figure}

\frenchspacing              % Better looking spacings after periods
\pagestyle{empty}           % No pagenumbers/headers/footers

%%% Custom sectioning (sectsty package)
%%% ------------------------------------------------------------
\usepackage{sectsty}

\sectionfont{%			            % Change font of \section command
	\usefont{OT1}{phv}{b}{n}%		% bch-b-n: CharterBT-Bold font
	\sectionrule{0pt}{0pt}{-5pt}{1pt}}

%%% Macros
%%% ------------------------------------------------------------
\newlength{\spacebox}
\settowidth{\spacebox}{8888888888}			% Box to align text
\newcommand{\sepspace}{\vspace*{1em}}		% Vertical space macro

\newcommand{\MyName}[1]{ % Name
		\Huge \usefont{OT1}{phv}{b}{n} \hfill #1
		\par \normalsize \normalfont}
		
\newcommand{\MySlogan}[1]{ % Slogan (optional)
		\large \usefont{OT1}{phv}{m}{n}\hfill \textit{#1}
		\par \normalsize \normalfont}

\newcommand{\NewPart}[1]{\section*{\uppercase{#1}}}

\newcommand{\PersonalEntry}[2]{
		\noindent\hangindent=2em\hangafter=0 % Indentation
		\parbox{\spacebox}{        % Box to align text
		\textit{#1}}		       % Entry name (birth, address, etc.)
		\hspace{8em} #2 \par}    % Entry value

\newcommand{\SkillsEntry}[2]{      % Same as \PersonalEntry
		\noindent\hangindent=2em\hangafter=0 % Indentation
		\parbox{\spacebox}{        % Box to align text
		\textit{#1}}			   % Entry name (birth, address, etc.)
		\hspace{1.5em} #2 \par}    % Entry value	
		
\newcommand{\EducationEntry}[4]{
		\noindent \textbf{#1} \hfill      % Study
		\colorbox{White}{%
			\parbox{7cm}{%
			\hfill\color{Black}#2}} \par  % Duration
		\noindent \textit{#3} \par        % School
		\noindent\hangindent=2em\hangafter=0 \small #4 % Description
		\normalsize \par}

\newcommand{\WorkEntry}[4]{				  % Same as \EducationEntry
		\noindent \textbf{#1} \hfill      % Jobname
		\noindent\colorbox{Black}{\color{White}#2} \par  % Duration
		\noindent \textit{#3} \par              % Company
		\noindent\hangindent=2em\hangafter=0 \small #4 % Description
		\normalsize \par}

%%% Begin Document
%%% ------------------------------------------------------------
\begin{document}

% you can upload a photo and include it here...
\begin{center}
	%\includegraphics[width=0.15\textwidth]{guise.jpg}
\end{center}


\MyName{Eduardo Cavero Guzmán}
\sepspace

%%% Personal details
%%% ------------------------------------------------------------
\NewPart{Datos personales}{}

\PersonalEntry{Dni}{40126410}
\PersonalEntry{Nacimiento}{7 de febrero, 1979}
\PersonalEntry{Dirección}{Condominio Real B7-304 Av. José Saco Rojas 1149 Carabayllo}
\PersonalEntry{Teléfono}{(+51) 989 374 805}
\PersonalEntry{E--Mail}{\url{ecavero@gmail.com}}


%%% Education
%%% ------------------------------------------------------------
\NewPart{Estudios}{}

\EducationEntry{Ingeniería de Sistemas e Informática}{1998 - presente}{Universidad Alas Peruanas}{}
\sepspace


%%% Work experience
%%% ------------------------------------------------------------
\NewPart{Experiencia laboral}{}

\EducationEntry{Analista de Sistemas}{Setiembre 2012 - Octubre 2019}{Grupo Vital}{
\begin{itemize}
\item{Parte del equipo de desarrollo del Sismedic.com \url{www.sismedic.com.pe} (sistema médico ocupacional web).}
\item{He desarrollado mostrar el audiograma en el examen de Audiometría, el cuadro comparativo e informe del examen en Word.}
\item{He desarrollado la comunicación entre el softaware del audiómetro con el módulo de audiometría para que carguen los datos del examen.}
\item{He desarrollado la comunicación entre el software del espirómetro con el módulo de espirometría.}
\item{He desarrollado la base para el reemplazo de etiquetas en los informes ocupacionales en Word. También he desarrollado la base para la impresión masiva de estos documentos en un solo archivo.}
\item{He apoyado en el desarrollo de una aplicación de escritorio en Java que permite a los auditores médicos imprimir los consentimientos, informes de exámenes e informes finales.}
\item{He desarrollado la comunicación entre el software del lector de firma y huella con el módulo de citas.}
\end{itemize}
}
\sepspace

\EducationEntry{Programador}{Mayo 2008 - Agosto 2012}{Universidad Ricardo Palma}{
\begin{itemize}
\item{Desarrollo y mantenimiento de los sistemas de información, especialmente el sistema de matrícula.}
\item{Desarrollé un módulo para la migración de los datos de los alumnos del sistema académico al Aula Virtual de la universidad. Las bases de datos fueron SQL Server 2000 y MySQL 5 respectivamente.}
\end{itemize}
}
\sepspace

\NewPart{Conocimientos}{}{}{
\begin{itemize}
\item{Inglés: Escrito Nativo}
\item{Inglés: Oral Nativo}
\item{EJB: Semi Senior}
\item{HTML: Semi Senior}
\item{JEE: Semi Senior}
\item{JSE: Senior}
\item{Java: Senior}
\item{JSP:  Senior}
\item{JSF/PrimeFaces: Semi Senior}
\item{MS T-SQL: Junior}
\item{Linux: Junior}
\item{PC Windows 7: Junior}

\end{itemize}
}

\NewPart{Temas de Interés}{}{}{
\begin{itemize}
\item{Groovy: Lenguaje de programación dinámica que corre en la JVM con el cual se pueden hacer scripts, aplicaciones de escritorio y aplicaciones web.}
\item{Grails: Framework para desarrollo de aplicaciones web basado en Groovy.}
\item{Test Driven Development (TDD): Práctica de Ingenería de Software que consiste en escribir las pruebas de una funcionalidad primero, luego implementar dicha funcionalidad y finalmente refactorizar.}
\item{Bitcoin: Moneda digital descentralizada y programable. Esto abrió el camino a lo que ahora se conoce como la tecnología Blockchain.}
\item{Docker: Una plataforma para que desarrolladores y administradores de sistemas puedan construir, desplegar y ejecutar aplicaciones distribuidas, ya sea en laptops, máquinas virtuales en producción o en la nube.}
\item{LaTex: Sistema de composición de textos, orientado a la creación de documentos. Es usado para la generación de artículos y libros científicos que incluyen, entre otros elementos, expresiones matemáticas.}
\end{itemize}
}

\end{document}

